\documentclass[a4paper,12pt, titlepage]{report}

\usepackage[utf8]{inputenc} % set input encoding (not needed with XeLaTeX

\usepackage{graphicx} % support the \includegraphics command and options

% \usepackage[parfill]{parskip} % Activate to begin paragraphs with an empty line rather than an indent

%%% PACKAGES
\usepackage{booktabs} % for much better looking tables
\usepackage{array} % for better arrays (eg matrices) in maths
\usepackage{paralist} % very flexible & customisable lists (eg. enumerate/itemize, etc.)
\usepackage{verbatim} % adds environment for commenting out blocks of text & for better verbatim
\usepackage{subfig} % make it possible to include more than one captioned figure/table in a single float
\usepackage{polski}
\usepackage[utf8]{inputenc}
\usepackage[polish]{babel}
\usepackage[a4paper,width=150mm,top=25mm,bottom=25mm]{geometry}
\usepackage{pythonhighlight}
% These packages are all incorporated in the memoir class to one degree or another...

%%% HEADERS & FOOTERS
\usepackage{fancyhdr} % This should be set AFTER setting up the page geometry
\pagestyle{fancy} % options: empty , plain , fancy
\renewcommand{\headrulewidth}{0pt} % customise the layout...
\lhead{}\chead{}\rhead{}
\lfoot{}\cfoot{\thepage}\rfoot{}
\fancyfoot{}
\fancyfoot[LE,RO]{\thepage}
\fancyfoot[LO,CE]{Chapter \thechapter}
\fancyfoot[CO,RE]{Author Name}

%%% SECTION TITLE APPEARANCE
\usepackage{sectsty}
\allsectionsfont{\sffamily\mdseries\upshape} % (See the fntguide.pdf for font help)
% (This matches ConTeXt defaults)

%%% ToC (table of contents) APPEARANCE
\usepackage[nottoc,notlof,notlot]{tocbibind} % Put the bibliography in the ToC
\usepackage[titles,subfigure]{tocloft} % Alter the style of the Table of Contents
\renewcommand{\cftsecfont}{\rmfamily\mdseries\upshape}
\renewcommand{\cftsecpagefont}{\rmfamily\mdseries\upshape} % No bold!

\graphicspath{ {../ExEffects/} }

%%% END Article customizations

%%% The "real" document content comes below...

\def\uczelnia{{Uniwersytet Kardynała Stefana Wyszyńskiego w~Warszawie}\\
Wydział Matematyczno-Przyrodniczy \\ Szkoła Nauk Ścisłych}
\def\nralbumu{107418}
\title{Przetwarzanie obrazów\\Sprawozdanie z laboratorium}
\author{Małgorzata Wiśniewska}
\date{Warszawa, 2020} 

\begin{document}
\maketitle
\tableofcontents
\chapter{Wstęp}
\section{Format obrazów}
\section{Instrukcja obsługi programu}

\chapter{Operacje ujednolicania obrazów}
Operacje ujednolicania obrazów dzieli się na dwa etapy. Pierwszym etapem jest ujednolicanie geometryczne, drugim jest ujednolicenie rozdzielczościowe. W prezentowanym programie ujednolicane są dwa obrazy, w taki sposób, że mniejszy z nich jest doprowadzany do takiego samego rozmiaru jak większy. Skutkuje to wygenerowaniem nowego obrazu o zwiększonej ilości piksli niż początkowa wartość. Dzięki zastosowaniu tego typu ujednolicania w efekcie nie następuje widoczny spadek jakości. 
\section{Ujednolicanie obrazów szarych geometryczne}
\textbf{Opis algorytmu}
\par Operacje geometrycznego ujednolicania polega na wyrównaniu liczby piksli w kolumnach i wierszach w obu obrazach, poprzez zwiększenie liczby piksli w kolumnach i wierszach mniejszego z obrazów.
\begin{enumerate}
\item Wybierz największą wysokość i największą szerokość spośród obu obrazów.
\item Jeśli dany obraz ma mniejszą wysokość lub szerokość, wypełnij różnicę pikslami o wartości 1, tak, żeby wysokość i szerokość obu obrazów była równa.
\end{enumerate}
\begin{figure}%
    \centering
    \subfloat[label 1]{{\includegraphics[width=5cm]{img1} }}%
    \qquad
    \subfloat[label 2]{{\includegraphics[width=5cm]{img2} }}%
    \caption{2 Figures side by side}%
    \label{fig:example}%
\end{figure}
\section{Ujednolicanie obrazów szarych rozdzielczościowe}
\section{Ujednolicanie obrazów RGB geometryczne}
\section{Ujednolicanie obrazów RGB rozdzielczościowe}

\chapter{Operacje sumowania arytmetycznego obrazów szarych}
\section{Sumowanie obrazów szarych}
\subsection{Sumowanie obrazu z określoną stałą}
\subsection{Sumowanie dwóch obrazów}
\section{Mnożenie obrazów szarych}
\subsection{Mnożenie obrazu przez określoną stałą}
\subsection{Mnożenieobrazu przez inny obraz}
\section{Mieszanie obrazów z określonym współczynnikiem}
\section{Potęgowanie obrazu z zadaną potęgą}
\section{Dzielenie obrazów szarych}
\subsection{Dzielenie obrazu przez zadaną stałą}
\subsection{Dzielenie obrazu przez inny obraz}
\section{Pierwiastkowanie obrazu}
\section{Logarytmowanie obrazu}

\chapter{Operacje sumowania arytmetycznego obrazów barwowych}
\section{Sumowanie obrazów barwowych}
\subsection{Sumowanie obrazu z określoną stałą}
\subsection{Sumowanie dwóch obrazów}
\section{Mnożenie obrazów barwowych}
\subsection{Mnożenie obrazu przez określoną stałą}
\subsection{Mnożenieobrazu przez inny obraz}
\section{Mieszanie obrazów z określonym współczynnikiem}
\section{Potęgowanie obrazu z zadaną potęgą}
\section{Dzielenie obrazów barwowych}
\subsection{Dzielenie obrazu przez zadaną stałą}
\subsection{Dzielenie obrazu przez inny obraz}
\section{Pierwiastkowanie obrazu}
\section{Logarytmowanie obrazu}

\chapter{Operacje geometryczne na obrazie}
\section{Przemieszczanie obrazu o zadany wektor}
\section{Skalowanie obrazu}
\subsection{Skalowanie jednorodne}
\subsection{Skalowanie niejednorodne}
\section{Obracanie obrazu o dowolny kąt}
\section{Symetrie obrazu}
\subsection{Symetra względem osi OX}
\subsection{Symetria względem osi OY}
\subsection{Symetria względem zadanej prostej}
\section{Wycinanie fragmentów obrazów}
\section{Kopiowanie fragmentów obrazów}

\chapter{Operacje na histogramie obrazu szarego}
\section{Obliczanie histogramu}
\section{Przemieszczanie histogramu}
\section{Rozciąganie histogramu}
\section{Progowanie lokalne}
\section{Progowanie globalne}

\chapter{Operacje na histogramie obrazu barwowego}
\section{Obliczanie histogramu}
\section{Przemieszczanie histogramu}
\section{Rozciąganie histogramu}
\section{Progowanie 1 progowe lokalne}
\section{Progowanie 1 progowe globalne}
\section{Progowanie wieloprogowe lokalne}
\section{Progowanie wieloprogowe globalne}

\end{document}
